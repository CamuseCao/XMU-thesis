% !TeX TXS-program:compile = txs:///latexmk/{}[-pdfxe -synctex=1 -interaction=nonstopmode -silent -outdir=Temp %.tex] 
%
\documentclass[degree=doctor,bibstyle=numerical,font=empty]{Settings/XMUthesis}% numerical authoryear ,nottoc,notabs,notbib,undergraduate
% \documentclass[bibstyle=numbers,font=advance]{Settings/XMUthesis}

\XMUsetup{
    author                  = 酸奶                        ,
    title                   = 你的论文题目                ,
    date                    = \today                      , % 二〇一九年二月二十八日
    class                   = 2015级                      ,
    studentnumber           = 1972015xxxxxx               ,
    department              = 物理科学与技术学院          ,
    major                   = 微电子科学与工程            ,
    advisor                 = 校内指导老师 \quad 职称     ,
    otheradvisor            = 校外指导老师 \quad 职称     ,
    team                    = 物理科学与技术学院~理论物理 ,
    fundteam                = 物理科学与技术学院~理论物理 ,
    degree                  = 本\quad 科                  ,
    englishtitle            = Your English Title          ,
    majorordouble           = 主修                        , % 辅修
    lab                     = 实验室                      ,
    % 以下几项本科生无需填写,也不用删除
    classified_code         =  1234                       , % 分类号
	security_classification =  公开                       , % 内部 秘密 机密 公开 等
	UDC                     =  5                          , % Universal Decimal Classification
	submit_date             =  2020 年 8 月 10 日         , % 论文提交日期
	defense_date            =  2020 年 8 月 11 日         , % 论文答辩日期
	conferred_date          =  2020 年 8 月 21 日         , % 学位授予日期
	chairman                =  张三                       , % 答辩委员会主席
	referee                 =  李四                       , % 评阅人
}

\usepackage{Settings/XMU-logo}
\listfiles

\begin{document}

\maketitle

% !TeX root = ../XMU.tex

\chapter*{致谢} 

致谢语应以简短的文字对课题研究与论文撰写过程中曾直接给予帮助的人员(例如指导教师、答疑教师及其他人员)表示自己的谢意。
	
作为毕业论文提交时,应注意事项:致谢内容用小四号宋体。根据2016年2月施行的《厦门大学本科毕业论文(设计)规范》,致谢被放在论文起首。致谢结构一般分为三个部分:1,回顾;2,感谢; 3,承担责任以及献辞。第一部分可以简述论文写作的经历,所面对的挑战以及你如何应对。第二部分具体感谢在论文过程中给与你帮助的人。第三部分指出你将为自己的论文承担责任,如果你希望将此论文献给谁,可以在最后指出。致谢内容请亲自撰写,使其具备你个人的特色。抄袭任何模板内容是极其懒惰、没有意义、不负责任和错误的行为。
	

% !TeX root = ../Main/XMU.tex
\chapter*{摘要}

中文摘要。

摘要应具有独立性和自含性,语言精炼、明确,高度概括论文内容,以 400 字左右为宜。关键词应体现论文特色,具有语义性,在论文中有明确出处,以 3—5 个为宜。关键词另起一行排在摘要的下方,每个关键词之间用中文分号“;”分开,最后一个关键词不打标点符号。
	
	
\keywords{关键词1;关键词2;关键词3}	

\clearpage

\chapter*{Abstract}

English abstract . 英文摘要、关键词内容与中文相同,每个关键词之间用英文
分号“;”加一空格分开,最后一个关键词不打标点符号。中、
英文摘要及其关键词各置一页内。

\englishkeywords{keyword1;keyword2;keyword3}

\cleardoublepage
\xmutableofcontents
\pagestyle{fancy}
% !TeX root = ../Main/XMU.tex
\chapter{模板的使用说明}{The usage guide of this template}

\section{使用的前提}{Prediction}

为了使用该模板,需要安装一个TeX的发行版本。可以选择Texlive或者Miktex,他们都是跨平台的。而Texlive打包了比较多的宏包,较为庞大,Miktex则是临时下载没有的宏包。这里我推荐使用Miktex。但是对于Mac,推荐使用MacTeX,它是为Mac定制的发行版本,应该比较合适。特别提醒CTeX套装无法运行该模板。关于编译方式需选择XeLaTeX,否则无法正常编译该模板。


\section{几点说明}{Some notes}

为了正确使用该模板,请按照提示安装好可使用的TeX发行版本。因为论文内容比较多,因此采取了分文件的方式来构成该文档。主文档XMU.tex的位置位于Main下,正确编译后所得的pdf文件就在这里。Figure文件夹是存放图片的文件夹,该文件夹已经加入图片文件夹的位置,插入图片是无需多加路径,直接用文件名即可。关于Setting文件夹只需要把里面的Information.tex正确填入即可。而你需要编辑的仅有Body文件夹下的文件。

该模板是在厦门大学博士学位论文模板的基础上修改得到的,因为本科论文与博士学位论文的要求差别比较的,所以定制了该模板。由于本人水平有限,因此该模板写的并不好,但是应该勉强能够满足毕业论文的要求。但是仍然可能有许多错误的地方,希望各位使用者如果能发现错误之处能够提出。可以给我法邮件或者直接在github上面提issue。欢迎大家的参与,共同完善母校的模板。

由于本人是一名理科生,对文科的同学毕业论文的额外需求可能了解不多。虽说文科生用这个模板的可能性比较小,如果有人用,有额外的需求也可以提出。

联系方式:
邮箱: \href{mailto:camusecao@gmail.com}{camusecao@gmail.com}

github项目的地址 : \href{https://github.com/CamuseCao/XMU-Undergraduate-thesis-template}{XMU-Undergraduate-thesis-template}

% !TeX root = ../Main/XMU.tex
\chapter{模板的使用说明}{The usage guide of this template}

\section{使用的前提}{Prediction}

为了使用该模板,需要安装一个TeX的发行版本。可以选择Texlive或者Miktex,他们都是跨平台的。而Texlive打包了比较多的宏包,较为庞大,Miktex则是临时下载没有的宏包。这里我推荐使用Miktex。但是对于Mac,推荐使用MacTeX,它是为Mac定制的发行版本,应该比较合适。特别提醒CTeX套装无法运行该模板。关于编译方式需选择XeLaTeX,否则无法正常编译该模板。


\section{几点说明}{Some notes}

该模板是在厦门大学博士学位论文模板的基础上修改得到的,因为本科论文与博士学位论文的要求差别比较的,所以定制了该模板。由于本人水平有限,因此该模板写的并不好,但是应该勉强能够满足毕业论文的要求。但是仍然可能有许多错误的地方,希望各位使用者如果能发现错误之处能够提出。可以给我法邮件或者直接在github上面提issue。欢迎大家的参与,共同完善母校的模板。

由于本人是一名理科生,对文科的同学毕业论文的额外需求可能了解不多。虽说文科生用这个模板的可能性比较小,如果有人用,有额外的需求也可以提出。

我的邮箱是: camusecao@gmail.com

github项目的地址是 : 

% \nocite{*} 
\bibliography{Body/Reference}

\appendix
\chapter{代码}{Codes}
\section{代码实现}{Code implementation}
\subsection{画图}{Plot}
下面是一个Matlab的代码的插入,还可以插入其它类型的代码。有额外需求可以添加。

\lstinputlisting[language=Matlab]{../Body/Plot.m}

附录里面还可以放其它需要的内容,它们是文章的补充。


\chapter{字体}{Font}
\showfont
\backmatter
\chapter{校徽}{School badge}
\begin{figure}[htbp!]
\centering
\caption{厦大校徽}
\xmulogo[0.75]
\end{figure}
\begin{figure}[htbp!]
\centering
\caption{厦门大学}
\xmulogon[0.75]
\end{figure}
\end{document}